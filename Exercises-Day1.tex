\part{Day 1}
%% \phantomsection
\addcontentsline{toc}{section}{Advanced Rice Customization Patterns}
{\setlength{\baselineskip}%
  {0.0\baselineskip}
  \section*{\flushright Exercise 1\\Advanced Rice Customization Patterns}
  \hrulefill \par}

\addcontentsline{toc}{subsection}{Description}
\subsection*{Description}

\addcontentsline{toc}{subsection}{Goals}
\subsection*{Goals}
\begin{itemize}
  \item Create a custom module.
  \item Create a custom DD Control class.
  \item See interesting ways to customize the portal.
  \item Create a custom JSP/JSTL function.
  \item Create a custom JstlListener and see how that works.
  \item Create custom parameters for your application-config.xml
    configuration
  \item Add an extension to a business object
\end{itemize}

\addcontentsline{toc}{subsection}{Creating a Custom Module}
\subsection*{1 Creating a Custom Module}

\subsection*{1.1 Browse the Rice Source Project}

\subsection*{1.2 Create a New Directory}
%% eclipse screenshot here
Create a new directory in \emph{rice-src} called \emph{training}
\begin{lstlisting}[language=bash,backgroundcolor=\color{ubergray},caption={Directory creation for Linux
    users},frame=single,breaklines=true]
  mkdir training
\end{lstlisting}

\subsection*{1.3 Add the Standard Structure}
%% eclipse screenshot here
\begin{lstlisting}[language=bash,backgroundcolor=\color{ubergray},caption={Directory creation for Linux
    users},frame=single,breaklines=true]
  mkdir -p src/main/java/
  mkdir -p src/main/resources/
  mkdir -p src/main/config/
  mkdir -p src/main/webapp/
\end{lstlisting}

\subsection*{1.4 Create a pom.xml for the new Module}

\subsection*{1.5 Add a Base Package}
%% eclipse screenshot here

\subsection*{1.6 Create Base Spring Beans}
\subsubsection*{1.6.1 Stub out
  src/main/resources/training/SpringModuleBeans.xml}
\begin{lstlisting}[language=xml,backgroundcolor=\color{ubergray},caption={Base
  SpringModuleBeans.xml},frame=single,breaklines=true]
<beans xmlns="http://www.springframework.org/schema/beans" xmlns:xsi="http://www.w3.org/2001/XMLSchema-instance" xmlns:aop="http://www.springframework.org/schema/aop" xmlns:tx="http://www.springframework.org/schema/tx"
	xsi:schemaLocation="http://www.springframework.org/schema/beans
                           http://www.springframework.org/schema/beans/spring-beans-2.0.xsd
                           http://www.springframework.org/schema/tx
                           http://www.springframework.org/schema/tx/spring-tx-2.0.xsd
                           http://www.springframework.org/schema/aop
                           http://www.springframework.org/schema/aop/spring-aop-2.0.xsd">
</beans>
\end{lstlisting}

\subsubsection*{1.6.1 Add the module bean with a parent}
\begin{lstlisting}[language=xml,backgroundcolor=\color{ubergray},caption={Base
  SpringBeans.xml},frame=single,breaklines=true]

  <bean id="bookstoreModuleCconfiguration"
    parent="bookstoreModuleConfiguration-parentBean" />

  <bean id="bookstoreModuleConfiguration-parentBean" abstract="true"
     class="org.kuali.rice.kns.bo.ModuleConfiguration">
    <property name="namespaceCode" value="bookstore"/>
    <property name="initializeDataDictionary" value="true"/>
    <property name="dataDictionaryPackages">
      <list>
        <value>classpath:train/bookstore/bo/datadictionary/</value>
      </list>
    </property>
    <property name="databaseRepositoryFilePaths">
      <list>
	    <value>OJB-repository-bookstore.xml</value>
      </list>
    </property>
    <property name="packagePrefixes">
      <list>
        <value>train.bookstore.bo</value>
      </list>
    </property>
  </bean>
\end{lstlisting}

\subsubsection*{1.6.1 Add Spring datasource Configuration}
\begin{lstlisting}[language=xml,backgroundcolor=\color{ubergray},caption={Spring
    datasource setup},frame=single,breaklines=true]
  <bean id="trainingDatasource" class="org.kuali.rice.core.database.PrimaryDataSourceFactoryBean" lazy-init="true">
    <property name="preferredDataSourceParams">
      <list>
        <value>training.datasource</value>
      </list>
    </property>
    <property name="preferredDataSourceJndiParams">
      <list>
        <value>training.datasource.jndi.location</value>
      </list>
    </property>
    <property name="serverDataSource" value="false"/>
  </bean>

  <bean id="trainingOjbConfigurer" class="org.kuali.rice.core.ojb.BaseOjbConfigurer">
    <property name="jcdAliases">
      <list>
        <value>trainingDataSource</value>
      </list>
    </property>
    <property name="metadataLocation" value="classpath:training/OJB-repository-bookstore.xml" />
  </bean>
\end{lstlisting}

\subsubsection*{1.6.1 Setup platformAwareDao}
\begin{lstlisting}[language=xml,backgroundcolor=\color{ubergray},caption={Spring
    datasource setup src/main/resources/training/OJB-repository-bookstore.xml},frame=single,breaklines=true]
  <bean id="platformAwareDao" abstract="true" class="org.kuali.rice.kns.dao.impl.PlatformAwareDaoBaseOjb">
    <property name="jcdAlias" value="trainingDataSource" />
    <property name="dbPlatform" ref="dbPlatform" />
  </bean>
\end{lstlisting}

\subsection*{1.7 Stub Base OJB Mapping}
\begin{lstlisting}[language=xml,backgroundcolor=\color{ubergray},caption={Stubbed
  OJB Descriptor file src/main/resources/OJB-repository-training.xml},frame=single,breaklines=true]
<descriptor-repository version="1.0">

  <jdbc-connection-descriptor jcd-alias="trainingDataSource" default-connection="false" jdbc-level="3.0" eager-release="false" batch-mode="false"
      useAutoCommit="0" ignoreAutoCommitExceptions="false">
    <sequence-manager className="org.kuali.rice.core.ojb.ConfigurableSequenceManager">
      <attribute attribute-name="property.prefix" attribute-value="datasource.ojb.sequenceManager" />
    </sequence-manager>
    <object-cache class="org.apache.ojb.broker.cache.ObjectCachePerBrokerImpl" />
  </jdbc-connection-descriptor>
</descriptor>
\end{lstlisting}

\addcontentsline{toc}{subsection}{Creating a Custom Data Dictionary Control}
\subsection*{2 Creating a Custom Data Dictionary Control}
Control definitions are just Spring beans accessed via the
BusinessObjectMetadataService and the DataDictionaryService. Besides
controls, you can create or extend any aspect of the DataDictionary
including relationships, lookup definitions, and the workflow
attributes.

The core DataDictionary classes are located in
\emph{impl/src/main/java/org/kuali/rice/kns/datadictionary/} of your
rice source code distribution.
%% Show eclipse screenshot here

\subsubsection*{2.1 Add a datadictionary Package to Our Module}
%% Add an eclipse screenshot

\begin{lstlisting}[language=bash,backgroundcolor=\color{ubergray},caption={Directory creation for Linux
    users},frame=single,breaklines=true]
  mkdir -p training/src/main/java/training/datadictionary/control
\end{lstlisting}

\subsubsection*{2.2 Stub SuggestsBox Control in src/main/java/training/datadictionary/control}
\begin{lstlisting}[language=xml,backgroundcolor=\color{ubergray},caption={Stubbed
  OJB Descriptor file src/main/resources/OJB-repository-training.xml},frame=single,breaklines=true]
package training.datadictionary.control;

/**
 *
 */
public class SuggestsBoxDefinition extends ControlDefinitionBase {
    private static final long serialVersionUID = -1L;

	public SuggestsBoxDefinition() {
    }

    /**
     * @see java.lang.Object#toString()
     */
    public String toString() {
        return "SuggestsBoxDefinition";
    }
}
\end{lstlisting}

\addcontentsline{toc}{subsection}{Creating a Custom JSTL Function}
\subsection*{3 Creating a Custom JSTL Function}
\subsubsection*{3.1 Add a web Package to Our Module}
%% Add an eclipse screenshot

\begin{lstlisting}[language=bash,backgroundcolor=\color{ubergray},caption={Directory creation for Linux
    users},frame=single,breaklines=true]
  mkdir -p training/src/main/java/training/web/
\end{lstlisting}



{\setlength{\baselineskip}%
  {0.0\baselineskip}
  \section*{\flushright Exercise 2\\Portal Customization}
  \hrulefill \par}

\subsection*{Description}

\subsection*{Goals}
\begin{itemize}
  \item Learn new ways to use javascript to communicate with SOA 
  \item How to creatively add functionality and rich user interface
    design to the Kuali Portal
\end{itemize}


\newpage
  {\setlength{\baselineskip}%
           {0.0\baselineskip}
  \section*{Notes}
  \hrulefill \par}

