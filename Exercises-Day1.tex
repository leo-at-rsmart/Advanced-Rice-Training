\part{Day 1}
%% \phantomsection
\addcontentsline{toc}{section}{Advanced Rice Customization Patterns}
{\setlength{\baselineskip}%
  {0.0\baselineskip}
  \section*{\flushright Exercise 1\\Advanced Rice Customization Patterns}
  \hrulefill \par}

\addcontentsline{toc}{subsection}{Description}
\subsection*{Description}

\addcontentsline{toc}{subsection}{Goals}
\subsection*{Goals}
\begin{itemize}
  \item Create a custom module.
  \item Create a custom DD Control class.
  \item See interesting ways to customize the portal.
  \item Create a custom JSP/JSTL function.
  \item Create a custom JstlListener and see how that works.
  \item Create custom parameters for your application-config.xml
    configuration
  \item Add an extension to a business object
\end{itemize}

\addcontentsline{toc}{subsection}{Creating a Custom Module}
\subsection*{1 Creating a Custom Module}

\subsection*{1.1 Browse the Rice Source Project}

\subsection*{1.2 Create a New Directory}
%% eclipse screenshot here
Create a new directory in \emph{rice-src} called \emph{training}
\begin{lstlisting}[language=bash,basicstyle=\scriptsize,backgroundcolor=\color{ubergray},caption={Directory
    creation for Linux users},frame=single,breaklines=true]
  mkdir training
\end{lstlisting}

\subsection*{1.3 Add the Standard Structure}
%% eclipse screenshot here
\begin{lstlisting}[language=bash,basicstyle=\scriptsize,backgroundcolor=\color{ubergray},caption={Directory creation for Linux
    users},frame=single,breaklines=true]
  mkdir -p src/main/java/
  mkdir -p src/main/resources/
  mkdir -p src/main/config/
  mkdir -p src/main/webapp/
\end{lstlisting}

\subsection*{1.4 Create a pom.xml for the new Module}

\subsection*{1.5 Add a Base Package}
%% eclipse screenshot here

\subsection*{1.6 Create Base Spring Beans}
\subsubsection*{1.6.1 Stub out
  src/main/resources/training/SpringModuleBeans.xml}
\begin{lstlisting}[numbers=left,language=xml,basicstyle=\scriptsize,backgroundcolor=\color{ubergray},caption={Base
  SpringModuleBeans.xml},frame=single,breaklines=true]
<beans xmlns="http://www.springframework.org/schema/beans" xmlns:xsi="http://www.w3.org/2001/XMLSchema-instance" xmlns:aop="http://www.springframework.org/schema/aop" xmlns:tx="http://www.springframework.org/schema/tx"
	xsi:schemaLocation="http://www.springframework.org/schema/beans
                           http://www.springframework.org/schema/beans/spring-beans-2.0.xsd
                           http://www.springframework.org/schema/tx
                           http://www.springframework.org/schema/tx/spring-tx-2.0.xsd
                           http://www.springframework.org/schema/aop
                           http://www.springframework.org/schema/aop/spring-aop-2.0.xsd">
</beans>
\end{lstlisting}

\subsubsection*{1.6.1 Add the module bean with a parent}
\begin{lstlisting}[numbers=left,language=xml,basicstyle=\scriptsize,backgroundcolor=\color{ubergray},caption={Base
  SpringBeans.xml},frame=single,breaklines=true]

  <bean id="bookstoreModuleCconfiguration"
    parent="bookstoreModuleConfiguration-parentBean" />

  <bean id="bookstoreModuleConfiguration-parentBean" abstract="true"
     class="org.kuali.rice.kns.bo.ModuleConfiguration">
    <property name="namespaceCode" value="bookstore"/>
    <property name="initializeDataDictionary" value="true"/>
    <property name="dataDictionaryPackages">
      <list>
        <value>classpath:train/bookstore/bo/datadictionary/</value>
      </list>
    </property>
    <property name="databaseRepositoryFilePaths">
      <list>
	    <value>OJB-repository-bookstore.xml</value>
      </list>
    </property>
    <property name="packagePrefixes">
      <list>
        <value>train.bookstore.bo</value>
      </list>
    </property>
  </bean>
\end{lstlisting}

\subsubsection*{1.6.1 Add Spring datasource Configuration}
\begin{lstlisting}[basicstyle=\scriptsize,numbers=left,language=xml,backgroundcolor=\color{ubergray},caption={Spring
    datasource setup},frame=single,breaklines=true]
  <bean id="trainingDatasource" class="org.kuali.rice.core.database.PrimaryDataSourceFactoryBean" lazy-init="true">
    <property name="preferredDataSourceParams">
      <list>
        <value>training.datasource</value>
      </list>
    </property>
    <property name="preferredDataSourceJndiParams">
      <list>
        <value>training.datasource.jndi.location</value>
      </list>
    </property>
    <property name="serverDataSource" value="false"/>
  </bean>

  <bean id="trainingOjbConfigurer" class="org.kuali.rice.core.ojb.BaseOjbConfigurer">
    <property name="jcdAliases">
      <list>
        <value>trainingDataSource</value>
      </list>
    </property>
    <property name="metadataLocation" value="classpath:training/OJB-repository-bookstore.xml" />
  </bean>
\end{lstlisting}

\subsubsection*{1.6.1 Setup platformAwareDao}
\begin{lstlisting}[basicstyle=\scriptsize,numbers=left,language=xml,backgroundcolor=\color{ubergray},caption={Spring
    datasource setup src/main/resources/training/OJB-repository-bookstore.xml},frame=single,breaklines=true]
  <bean id="platformAwareDao" abstract="true" class="org.kuali.rice.kns.dao.impl.PlatformAwareDaoBaseOjb">
    <property name="jcdAlias" value="trainingDataSource" />
    <property name="dbPlatform" ref="dbPlatform" />
  </bean>
\end{lstlisting}

\subsection*{1.7 Stub Base OJB Mapping}
\begin{lstlisting}[basicstyle=\scriptsize,numbers=left,language=xml,backgroundcolor=\color{ubergray},caption={Stubbed
  OJB Descriptor file src/main/resources/OJB-repository-training.xml},frame=single,breaklines=true]
<descriptor-repository version="1.0">

  <jdbc-connection-descriptor jcd-alias="trainingDataSource" default-connection="false" jdbc-level="3.0" eager-release="false" batch-mode="false"
      useAutoCommit="0" ignoreAutoCommitExceptions="false">
    <sequence-manager className="org.kuali.rice.core.ojb.ConfigurableSequenceManager">
      <attribute attribute-name="property.prefix" attribute-value="datasource.ojb.sequenceManager" />
    </sequence-manager>
    <object-cache class="org.apache.ojb.broker.cache.ObjectCachePerBrokerImpl" />
  </jdbc-connection-descriptor>
</descriptor>
\end{lstlisting}

\subsection*{1.8 Stub Module struts-config.xml}

\subsection*{1.9 Add struts-config.xml to Rice Project web.xml}

\addcontentsline{toc}{subsection}{Creating a Custom Data Dictionary Control}
\subsection*{2 Creating a Custom Data Dictionary Control}
Control definitions are just Spring beans accessed via the
BusinessObjectMetadataService and the DataDictionaryService. Besides
controls, you can create or extend any aspect of the DataDictionary
including relationships, lookup definitions, and the workflow
attributes.

The core DataDictionary classes are located in
\emph{impl/src/main/java/org/kuali/rice/kns/datadictionary/} of your
rice source code distribution.
%% Show eclipse screenshot here

\subsubsection*{2.1 Add a datadictionary Package to Our Module}
%% Add an eclipse screenshot

\begin{lstlisting}[basicstyle=\scriptsize,language=bash,backgroundcolor=\color{ubergray},caption={Directory creation for Linux
    users},frame=single,breaklines=true]
  mkdir -p training/src/main/java/training/datadictionary/control
\end{lstlisting}

\subsubsection*{2.2 Stub SuggestsBox Control in src/main/java/training/datadictionary/control}
\begin{lstlisting}[basicstyle=\scriptsize,numbers=left,language=xml,backgroundcolor=\color{ubergray},caption={Stubbed
  OJB Descriptor file src/main/resources/OJB-repository-training.xml},frame=single,breaklines=true]
package training.datadictionary.control;

/**
 *
 */
public class SuggestsBoxDefinition extends ControlDefinitionBase {
    private static final long serialVersionUID = -1L;

	public SuggestsBoxDefinition() {
    }

    /**
     * @see java.lang.Object#toString()
     */
    public String toString() {
        return "SuggestsBoxDefinition";
    }
}
\end{lstlisting}

\addcontentsline{toc}{subsection}{Creating a Custom JSTL Function}
\subsection*{3 Creating a Custom JSTL Function}
\subsubsection*{3.1 Add a web Package to Our Module}
%% Add an eclipse screenshot

\begin{lstlisting}[basicstyle=\scriptsize,language=bash,backgroundcolor=\color{ubergray},caption={Directory creation for Linux
    users},frame=single,breaklines=true]
  mkdir -p training/src/main/java/training/web/
\end{lstlisting}

\subsubsection*{3.2 Stub a TrainingFunctions Class in the web
  Package}
\begin{lstlisting}[basicstyle=\scriptsize,numbers=left,language=java,backgroundcolor=\color{ubergray},caption={training/web/TrainingFunctions},frame=single,breaklines=true]
package training.web;

/**
 * Full of static methods for JSTL function access.
 * 
 */
public final class TrainingFunctions {
}
\end{lstlisting}

\subsubsection*{3.2 Add a Method that Fetches System Parameters}
\begin{lstlisting}[basicstyle=\scriptsize,numbers=left,language=java,backgroundcolor=\color{ubergray},caption={training/web/TrainingFunctions},frame=single,breaklines=true]
...
...
import static org.kuali.rice.kns.service.KNSServiceLocator.getParameterService

public final static boolean paramExists(final String component, final String
name) {
    return getParameterService().parameterExists("KR-TRN", component, name);
}

public final static String getParameter(final String component, final String
name) {
    return getParameterService().getParameterValue("KR-TRN", component, name);
}

public final static List<String> getParameters(final String component, final String
name) {
    return getParameterService().getParameterValues("KR-TRN", component, name);
}

public final static boolean isEnabled(final String component, final String
name) {
    return getParameterService().getIndicatorParameter("KR-TRN", component, name);
}
\end{lstlisting}

\subsubsection*{3.3 Stub a Tag Library Definition for the Module}
\begin{lstlisting}[basicstyle=\scriptsize,numbers=left,language=xml,backgroundcolor=\color{ubergray},caption={src/main/webapp/WEB-INF/tlds/trnfunc.tld
  Tag Library Definition},frame=single,breaklines=true]
<taglib xmlns="http://java.sun.com/xml/ns/j2ee" xmlns:xsi="http://www.w3.org/2001/XMLSchema-instance" xsi:schemaLocation="http://java.sun.com/xml/ns/j2ee http://java.sun.com/xml/ns/j2ee/web-jsptaglibrary_2_0.xsd" version="2.0">
</taglib>
\end{lstlisting}

\subsubsection*{3.4 Define the Tag Library}
\begin{lstlisting}[basicstyle=\scriptsize,numbers=left,language=xml,backgroundcolor=\color{ubergray},caption={src/main/webapp/WEB-INF/tlds/trnfunc.tld
  Tag Library Definition},frame=single,breaklines=true]
    <description>Training functions library</description>
    <display-name>Training functions</display-name>
    <tlib-version>1.0</tlib-version>
    <short-name>fn</short-name>
    <uri>http://www.kuali.org/jsp/jstl/functions</uri>
\end{lstlisting}

\subsubsection*{3.5 Add the Functions from TrainingFunctions}
\begin{lstlisting}[basicstyle=\scriptsize,numbers=left,language=xml,backgroundcolor=\color{ubergray},caption={src/main/webapp/WEB-INF/tlds/trnfunc.tld
  Tag Library Definition},frame=single,breaklines=true]
    <function>
        <description>Access System Parameters from JSTL</description>
        <name>parameterExists</name>
        <function-class>training.web.TrainingFunctions</function-class>
        <function-signature>boolean parameterExists(java.lang.String, java.lang.String)</function-signature>
        <example>&lt;c:if
        test="${trn-fn:parameterExists('Document', 'ACTIVE_FILE_TYPES')}"&gt;&lt;/c:if&gt;</example>
    </function>

    <function>
        <description>Access System Parameters from JSTL</description>
        <name>getParameter</name>
        <function-class>training.web.TrainingFunctions</function-class>
        <function-signature>java.lang.String getParameter(java.lang.String, java.lang.String)</function-signature>
        <example>${trn-fn:getParameter('Document', 'ACTIVE_FILE_TYPES')}</example>
    </function>

    <function>
        <description>Access System Parameters from JSTL</description>
        <name>getParameters</name>
        <function-class>training.web.TrainingFunctions</function-class>
        <function-signature>java.util.List getParameters(java.lang.String, java.lang.String)</function-signature>
        <example>${trn-fn:getParameters('Document', 'ACTIVE_FILE_TYPES')}</example>
    </function>

    <function>
        <description>Access System Parameters from JSTL</description>
        <name>isEnabled</name>
        <function-class>training.web.TrainingFunctions</function-class>
        <function-signature>boolean isEnabled(java.lang.String, java.lang.String)</function-signature>
        <example>${trn-fn:isEnabled('Document', 'ALLOW_NEGATIVE_BALANCE_IND')}</example>
    </function>
\end{lstlisting}

\subsubsection*{3.6 Test Yourself: What have you learned?}
Using the same examples above, add a call to \emph{hasPermission} that
you can use to restrict a tab in the portal specifically for FO's.

\addcontentsline{toc}{subsection}{Creating a Custom JSTL Function}
\subsection*{3 Adding to JstlConstantsInitListener}
JstlConstantsInitListener can be found in
\emph{impl/src/main/java/org/kuali/rice/kns/web/listener/JstlConstantsInitListener.java}
of your rice source distribution.
%% Eclipse screenshot here

\subsubsection*{3.1 Add a listener Package to Our Module}
%% Add an eclipse screenshot

\begin{lstlisting}[basicstyle=\scriptsize,language=bash,backgroundcolor=\color{ubergray},caption={Directory creation for Linux
    users},frame=single,breaklines=true]
  mkdir -p training/src/main/java/training/web/listener
\end{lstlisting}

\subsubsection*{3.1 Stub out a ContextListener}
\begin{lstlisting}[basicstyle=\scriptsize,numbers=left,language=java,backgroundcolor=\color{ubergray},caption={training/web/TrainingFunctions},frame=single,breaklines=true]
package training.web.listener;

import javax.servlet.ServletContext;
import javax.servlet.ServletContextEvent;
import javax.servlet.ServletContextListener;


/**
 * This class is the JstlContants implementation of the ServletContextListener.
 */
public class JstlConstantsInitListener implements ServletContextListener {
}
\end{lstlisting}


\subsubsection*{3.1 Add a Constant}
\begin{lstlisting}[basicstyle=\scriptsize,numbers=left,language=java,backgroundcolor=\color{ubergray},caption={training/web/TrainingFunctions},frame=single,breaklines=true]
package training;


import org.kuali.rice.core.util.JSTLConstants;

public class TrainingConstants extends JSTLConstants {
  public static String CURRENT_PROCESS_DATE_PARAMETER = "CURRENT_PROCESS_DATE";
}
\end{lstlisting}

\subsubsection*{3.1 Add Constants to the Listener}
\begin{lstlisting}[basicstyle=\scriptsize,numbers=left,language=java,backgroundcolor=\color{ubergray},caption={training/web/TrainingFunctions},frame=single,breaklines=true]
    import training.TrainingConstants;
    ...
    ...
    public void contextInitialized(ServletContextEvent sce) {

        ServletContext context = sce.getServletContext();
        // publish application constants into JSP app context with name "Constants"
        context.setAttribute("TrainingConstants", new TrainingConstants());
    }
\end{lstlisting}

\subsubsection*{3.1 Add the New InitListener to the web.xml}
The listener needs to be added to the Rice web.xml file. There is only
one web.xml file for an application. Each rice project has one. It
will be located in your project source tree at
src/main/webapp/WEB-INF/
%% Show Eclipse screenshot
 
In it, you will see a section with listeners that looks like:
\begin{lstlisting}[basicstyle=\scriptsize,numbers=left,language=java,backgroundcolor=\color{ubergray},caption={src/main/webapp/WEB-INF/web.xml},frame=single,breaklines=true]
    <listener>
      <listener-class>org.kuali.rice.core.web.listener.StandaloneInitializeListener</listener-class>
    </listener>

    <listener>
      <listener-class>org.kuali.rice.kns.web.listener.JstlConstantsInitListener</listener-class>
    </listener>

    <listener>
      <listener-class>org.kuali.rice.kns.web.listener.KualiHttpSessionListener</listener-class>
    </listener>
\end{lstlisting}

Just add yours. Now when your application starts, you will be able to
access your constants from the JSP/JSTL.

{\setlength{\baselineskip}%
  {0.0\baselineskip}
  \section*{\flushright Exercise 2\\Portal Customization}
  \hrulefill \par}

\subsection*{Description}

\subsection*{Goals}
\begin{itemize}
  \item Learn new ways to use javascript to communicate with SOA 
  \item How to creatively add functionality and rich user interface
    design to the Kuali Portal
\end{itemize}


\newpage
  {\setlength{\baselineskip}%
           {0.0\baselineskip}
  \section*{Notes}
  \hrulefill \par}

